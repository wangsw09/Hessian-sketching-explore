%%%%%%%%%%%%%%%%%%%%%%%%%%%%%%%%%%%%%%%%%
% Beamer Presentation
% LaTeX Template
% Version 1.0 (10/11/12)
%
% This template has been downloaded from:
% http://www.LaTeXTemplates.com
%
% License:
% CC BY-NC-SA 3.0 (http://creativecommons.org/licenses/by-nc-sa/3.0/)
%
%%%%%%%%%%%%%%%%%%%%%%%%%%%%%%%%%%%%%%%%%

%----------------------------------------------------------------------------------------
%	PACKAGES AND THEMES
%----------------------------------------------------------------------------------------

\documentclass{beamer}

\mode<presentation> {

% The Beamer class comes with a number of default slide themes
% which change the colors and layouts of slides. Below this is a list
% of all the themes, uncomment each in turn to see what they look like.

%\usetheme{default}
%\usetheme{AnnArbor}
%\usetheme{Antibes}
%\usetheme{Bergen}
%\usetheme{Berkeley}
%\usetheme{Berlin}
%\usetheme{Boadilla}
%\usetheme{CambridgeUS}
%\usetheme{Copenhagen}
%\usetheme{Darmstadt}
%\usetheme{Dresden}
%\usetheme{Frankfurt}
%\usetheme{Goettingen}
%\usetheme{Hannover}
%\usetheme{Ilmenau}
%\usetheme{JuanLesPins}
%\usetheme{Luebeck}
\usetheme{Madrid}
%\usetheme{Malmoe}
%\usetheme{Marburg}
%\usetheme{Montpellier}
%\usetheme{PaloAlto}
%\usetheme{Pittsburgh}
%\usetheme{Rochester}
%\usetheme{Singapore}
%\usetheme{Szeged}
%\usetheme{Warsaw}

% As well as themes, the Beamer class has a number of color themes
% for any slide theme. Uncomment each of these in turn to see how it
% changes the colors of your current slide theme.

%\usecolortheme{albatross}
%\usecolortheme{beaver}
%\usecolortheme{beetle}
%\usecolortheme{crane}
%\usecolortheme{dolphin}
%\usecolortheme{dove}
%\usecolortheme{fly}
%\usecolortheme{lily}
%\usecolortheme{orchid}
%\usecolortheme{rose}
%\usecolortheme{seagull}
%\usecolortheme{seahorse}
%\usecolortheme{whale}
%\usecolortheme{wolverine}

%\setbeamertemplate{footline} % To remove the footer line in all slides uncomment this line
%\setbeamertemplate{footline}[page number] % To replace the footer line in all slides with a simple slide count uncomment this line

%\setbeamertemplate{navigation symbols}{} % To remove the navigation symbols from the bottom of all slides uncomment this line
}

\usepackage{graphicx} % Allows including images
\usepackage{booktabs} % Allows the use of \toprule, \midrule and \bottomrule in tables

%----------------------------------------------------------------------------------------
%	TITLE PAGE
%----------------------------------------------------------------------------------------

\title[Iterative Hessian Sketch]{Iterative Hessian Sketch for Constrained Least-Squares} 

\author{Lydia Hsu, Shuaiwen Wang} 

\date{\today} % Date, can be changed to a custom date

\begin{document}

\begin{frame}
\titlepage % Print the title page as the first slide
\end{frame}

\begin{frame}
    \frametitle{Problem description}

    \begin{itemize}
        \item<1-> Consider the \textbf{constrained least-squares}
            \begin{equation*}
                x_{ls} := \arg\min_{x \in C} \; \frac{1}{2} \|Ax-y\|^2_2
            \end{equation*}
            where \quad $C$: constraint; \quad $A \in \mathbb{R}^{n \times p}$: design matrix; \quad $y$: response.
        
        \item<2-> Issue: when $n$ large, computationally \textbf{expensive};
        \item<3-> Idea: ``sketch'' $A$ using $SA$;
        \item<4-> $S \in \mathbb{R}^{m \times n}$: sketching matrix; \quad
            $\mathbb{E} S^\top S = I_n$.
    \end{itemize}
\end{frame}

\begin{frame}
    \frametitle{Common sketching matrices}

    \begin{itemize}
        \item<1-> sub-Gaussian: $S_{ij} \sim_{iid} \frac{1}{\sqrt{m}}
            \mathcal{N}(0, 1)$; 
        \item<2-> random row sampling: randomly select $m$ rows from $I_n$
            (rescaled by $\sqrt{\frac{n}{m}}$);
        \item<3-> randomized orthogonal system: randomly select $m$ rows from
            an orthogonal matrix $H$ (rescaled by $\sqrt{\frac{n}{m}}$);
    \end{itemize}
\end{frame}

\begin{frame}
    \frametitle{Classical Sketch}
    
    \begin{itemize}
        \item<1-> Sketch not only $A$, but also $y$:
            \begin{equation*}
                x_{cs} := \arg\min_{x \in C} \; \frac{1}{2} \|SAx-Sy\|^2_2
           \end{equation*}
       \item<2-> Classical sketch is \textbf{sub-optimal}:
           assume a linear ground truth $y = Ax_* + w$:
           \begin{equation*}
               \|x_{cs} - x_*\| \sim \|x_{ls} - x_*\|
               \quad \Rightarrow \quad
               m = \Theta(n)
           \end{equation*}
       \item<3-> We expect the optimal method satisfies $m \approx O(p)$
           \textcolor{red}{Think about this part more carefully?};
    \end{itemize}

\end{frame}

\begin{frame}
    \frametitle{Hessian Sketch}
    \begin{itemize}
        \item<1-> Only sketch $A$, but not $y$:
            \begin{equation*}
                x_{hs} := \arg\min_{x\in C} \; \frac{1}{2} \| S Ax\|^2_2 - \langle A^\top y, x \rangle.
            \end{equation*}
        \item<2-> Still \textbf{sub-optimal} in the same sense;
        \item The above serves as the building block for an iterative method
    \end{itemize}
\end{frame}

%------------------------------------------------

\begin{frame}
\frametitle{Iterative Hessian Sketch}
\begin{itemize}
\item Idea: construct a new least-squares problem for which the optimal solution is $x^{LS} - x^t$. Then applying HS to the problem will produce a new $x^{t+1}$ whose distance to $x^{LS}$ has been reduced by a factor of $\rho$.
\item $\hat{u} := argmin_{u \in \{C-x^t\}} \{\frac{1}{2} \| Au\|^2_2 - \langle A^T( y-Ax^t), u \rangle \}.$\\
By construction, $\hat{u} = x^{LS} - x^t$.
\item Advantages: sample size efficiency and computational time saving
\end{itemize}
\end{frame}

%------------------------------------------------

\begin{frame}
\frametitle{Iterative Hessian Sketch Algorithm}
\begin{block}{Algorithm}
\begin{enumerate}
\item Initialize $x^0 = 0$.\\
\item For iterations $t=0,1,2,...N-1$, generate independent sketches $S^{t+1} \in \mathbb{R}^{mxn}$, and perform the updates \\
$x^{t+1} = argmin_{x\in C} \{\frac{1}{2m} \| S^{t+1} A (x-x^t)\|^2_2 - \langle A^T(y-Ax^t), x \rangle \}.$ \\
\item Return $\hat{x} = x^N$.  
\end{enumerate}
\end{block}

\end{frame}


%------------------------------------------------
\section{Simulation Study}
%------------------------------------------------

\begin{frame}
\end{frame}

%------------------------------------------------

\begin{frame}
\frametitle{References}
\footnotesize{
\begin{thebibliography}{99} 
\bibitem[Smith, 2012]{p1} Sarlos (2006)
\newblock Title of the publication
\newblock \emph{Journal Name} 12(3), 45 -- 678.
\end{thebibliography}
}
\end{frame}

%------------------------------------------------
\begin{frame}
\frametitle{Evaluation for Sketch Performance}
\begin{block}{Cost Approximation}
In terms of $f$-cost, the approximated $\tilde{x}$ is said to be $\epsilon$-optimal if \\
\vspace{.5cm}
$f(x^{LS}) \leq f(\tilde{x}) \leq (1+\epsilon)^2 f(x^{LS})$.
\end{block}

\begin{block}{Solution Approximation}
In terms of $f$-cost, the prediction norm defined as \\
\vspace{.5cm}
$\|\tilde{x} - x^{LS}\|_A := \frac{1}{\sqrt{n}} \|A(\tilde{x} - x^{LS})\|_2$.
\end{block}

\end{frame}


\begin{frame}
    \begin{center}
        \Large Thank you!
    \end{center}
\end{frame}

%----------------------------------------------------------------------------------------

\end{document}
